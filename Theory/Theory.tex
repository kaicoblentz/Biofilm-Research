\documentclass[letterpaper, twoside]{article}
\usepackage[utf8]{inputenc}
\usepackage{fullpage}
\usepackage{amssymb}
\usepackage{amsmath}
\usepackage{graphicx}

% Text
\newcommand{\etc}{etc.}
\newcommand{\eg}{e.g.}
\newcommand{\ie}{i.e.}

\title{Biofilm Model Theory}
\author{Austen, Takumi, Mark, Phil}
\date{}

\begin{document}
\maketitle
\abstract{Summary of theory used in biofilm model.}

\section{Overview}

\section{Tank Environment}
The environment is modeled in this function as a rectangular tank holding the bulk liquid, with the existence of an inflow and outflow which carries biomass and substrate into the bulk liquid. The concentrations rates of the biomass and substrate are modeled by the following ordinary differential equations
\begin{equation} \label{eq: BiomassEquation}
  \frac{dx}{dt} = \left(\mu(S) - \frac{Q}{V}\right) x + V_{\mathrm{det}} S_A X_b
\end{equation}
\begin{equation} \label{eq: SubstrateEquation}
  \frac{dS}{dt} = \frac{\mu(S) x}{Y_{xs}} + \frac{Q S_{in}}{V} - \frac{Q S}{V} - S_A \frac{D_{aq}}{L_L} (C_o-C_s)
\end{equation}

\section{Growth-rate $\mu$}

\section{Biofilm Diffusion}
Substrates that exist within the tank will diffuse into the biofilm.  The diffusion process is described by
\begin{equation} \label{eq:diffusion}
  \frac{d^2 S_b}{dz^2} = \frac{\mu(S_b) X_b}{Y_{xs} D_e}.
\end{equation}
This differential equation is, typically, non-linear due the growth-rate $\mu$.

\subsection{Discretization and Linearization}
To solve it we use the direct, finite-difference method, which leads to the following discretized equation
\begin{equation} \label{eq:diff_dis}
  \frac{ S_{b,i-1} - 2 S_{b,i} + S_{b,i+1}}{dz^2} = \frac{\mu(S_{b,i}) X_b}{Y_{xs} D_e},
\end{equation}
which is valid at all the interior grid points, \ie, for $i=2,3,\dots,N_z-1$. 
This non-linear equation is solved by linearizing and then iterating the solution from an initial guess until converged.
The iterations are denoted by a superscript, \ie, $S_{b}^{(k)}$.  With this notation Eq.~\ref{eq:diff_dis} becomes
\begin{equation} \label{eq:diff_dis_iter}
  \frac{ S_{b,i-1}^{(k)} - 2 S_{b,i}^{(k)} + S_{b,i+1}^{(k)}}{\Delta z^2} = \frac{\mu\left(S_{b,i}^{(k)}\right) X_b}{Y_{xs} D_e} =  g\left(S_{b,i}^{(k)}\right).
\end{equation}
where we have introduced $g$ as the right-hand-side of the equation.

To linearize this equation, we employ the Taylor series of $g$ about the previous iteration $S_{b,i}^{(k-1)}$ which is
\begin{equation}\label{eq:TaylorSeries}
  g\left(S_{b,i}^{(k)}\right) =   g\left(S_{b,i}^{(k-1)}\right) + \left( S_{b,i}^{(k)} - S_{b,i}^{(k-1)}\right) \left.\frac{d g}{d S_b}\right|_{S_{b,i}^{(k-1)}} + \dots
\end{equation}

Combining Eqs.~\ref{eq:diff_dis_iter} and~\ref{eq:TaylorSeries} and keeping only the linear terms in the Taylor series leads to
\begin{equation} \label{eq:diff_linear}
  \frac{ S_{b,i-1}^{(k)} - 2 S_{b,i}^{(k)} + S_{b,i+1}^{(k)}}{\Delta z^2} =  g\left(S_{b,i}^{(k-1)}\right) + \left( S_{b,i}^{(k)} - S_{b,i}^{(k-1)}\right) \left.\frac{d g}{d S_b}\right|_{S_{b,i}^{(k-1)}} 
\end{equation}
which is linear with-respect-to $S_{b,i}^{(k)}$ and can be rearranged to
\begin{equation}
  \label{eq:diff_final}
  -S_{b,i-1}^{(k)} + \left( 2 +\Delta z^2\left.\frac{d g}{d S_b}\right|_{S_{b,i}^{(k-1)}}\right) S_{b,i}^{(k)} - S_{b,i+1}^{(k)}
  = \Delta z^2\left( S_{b,i}^{(k-1)} \left.\frac{d g}{d S_b}\right|_{S_{b,i}^{(k-1)}} - g\left(S_{b,i}^{(k-1)}\right)\right) 
\end{equation}

The derivative $\left.\frac{d g}{d S_b}\right|_{S_{b,i}^{(k-1)}}$ needs to be approximated and we use
\begin{equation}
  \label{eq:dgds}
  \left.\frac{d g}{d S_b}\right|_{S_{b,i}^{(k-1)}} = \frac{g\left(S_{b,i}^+\right) - g\left(S_{b,i}^{-}\right)}{\Delta S}
\end{equation}
where
\begin{align*}
  S_{b,i}^+&=S_{b,i}^{(k-1)}+\delta \text{\quad and} \\
  S_{b,i}^-&=\max\left[S_{b,i}^{(k-1)}+\delta,0\right]
\end{align*}
and $\Delta S = S_{b,i}^+ - S_{b,i}^-$ and $\delta=1\times 10^{-3}$ is an specified constant.  The maximum on $S_{b,i}^-$ ensures the concentration remains non-negative.

\subsection{Boundary Conditions}
Eq.~\ref{eq:diff_final} for $i=2,3,\dots,N_z-1$ provides $N_z-2$ equations for $S_{b}^{(k)}$.  The remaining equations come from the boundary conditions.  At the bottom of the biofilm ($z=0$) there is a wall and a no-flux boundary condition is appropriate, \ie,
\begin{equation}
  \label{eq:BC1}
  \left.\frac{d S_b}{dz}\right|_{z=0}= \frac{S_2 - S_1}{\Delta z} =0.
\end{equation}

At the top of the biofilm the substrate is diffusing from the tank into the biofilm.  Depending on the conditions in the tank, \eg, how well it is mixed, the flux of substrate into the biofilm may be controlled by the diffusion through the liquid in the tank.  This leads to a flux-matching condition that can be written as
\begin{equation}
  \label{eq:BC2}
  D_e \left.\frac{d S_b}{dz}\right|_{z=L_f} = D_{\mathrm{aq}} \frac{S - S_b(L_f)}{L_L}.
\end{equation}
where a simple diffusion model through the liquid has been used.  Discretizing the derivative using a finite-difference operator leads to
\begin{equation}
  \label{eq:BC2_dis}
  D_e \frac{S_{b,N_z} - S_{b,N_z-1}}{\Delta z} = D_{\mathrm{aq}} \frac{S - S_{b,N_z}}{L_L}.
\end{equation}
Rearranging leads to
\begin{equation}
  \label{eq:BC2_dis2}
  \left(D_e L_l + D_{\mathrm{aq}} \Delta z\right) S_{b,N_z} - D_e L_l S_{b,N_z-1} = D_{\mathrm{aq}} \Delta z S
\end{equation}
which is a useful form because if $L_l=0$ it simplifies to $S_{b,N_z}=S$ as expected without dividing by zero.

\subsection{Solution of System of Equations}
The previous two section describe the equations used to solve the diffusion problem through
the biofilm and apply appropriate boundary conditions.
In summary, Eq.~\ref{eq:diff_linear} for $i=2,3,\dots,N_z-1$ provides $N_z-2$ equations with
Eq.~\ref{eq:BC1} and Eq.~\ref{eq:BC2_dis2} providing the two other equations
for a total of $N_z$ equations for the $N_z$ unknowns $S_{b,i}$ for $i=1,\dots,N_z$.

The $N_z$ equations are solved by iteratively solving for $S_b$ using the matrices
\begin{equation}
  \label{eq:system}
  \renewcommand*{\arraystretch}{1.5}
  \begin{bmatrix}
    1 &-1  &   &   &   & \\
    L &  D & U &  &   & \\
    ~ & L &  D & U &  & \\
    ~ & ~ & \ddots &  \ddots & \ddots & \\
    ~ & ~ & ~ & L &  D & U\\
    ~ & ~ & ~ & ~ &  C_1 & C_2  
  \end{bmatrix}
  \begin{bmatrix}
    S_{b,1}^{(k)}\\
    S_{b,2}^{(k)}\\
    S_{b,3}^{(k)}\\
   \vdots\\
    S_{b,N_z-1}^{(k)}\\
    S_{b,N_z}^{(k)}
  \end{bmatrix}
  =
    \begin{bmatrix}
   0\\
    R_{2}\\
    R_{3}\\
    \vdots\\
    R_{N_z-1}\\
    C_3
  \end{bmatrix}
\end{equation}
The first row comes from Eq.~\ref{eq:BC1}.
The second through $N_z-1$ rows are Eq.~\ref{eq:diff_final} written with $i=2,\dots,N_z-1$ and the constants are $L=U=-1$,
\begin{align*}
  D&=\left( 2 +\Delta z^2\left.\frac{d g}{d S_b}\right|_{S_{b,i}^{(k-1)}}\right)\text{\quad and}\\
  R_i&=\Delta z^2\left( S_{b,i}^{(k-1)} \left.\frac{d g}{d S_b}\right|_{S_{b,i}^{(k-1)}} - g\left(S_{b,i}^{(k-1)}\right)\right) 
\end{align*}
The last row is Eq.~\ref{eq:BC2_dis2} with $C_1=D_e L_l$, $C_2=\left(D_e L_l + D_{\mathrm{aq}} \Delta z\right)$, and $C_3=D_{\mathrm{aq}} \Delta z S$.

The right-hand-side depends $S_{b,i}^{(k-1)}$, which is the concentration at the previous iteration.  The solution process is started with a guess, e.g., $S_b=0$. Iteration continues until
\begin{equation*}
  \max\left| S_{b,i}^{(k)} - S_{b,i}^{(k-1)} \right| < \mathrm{tol},
\end{equation*}
for a specified tolerance $\mathrm{tol}$.

\end{document}